\documentclass[10pt]{beamer}

\usetheme{Madrid}

\usepackage{graphicx}

\usepackage{booktabs}

\title{Uncertainty from $b$ quark masses in Z pT spectrum}

\subtitle{Status summary}

\author{AlphaS from Z pT analysis team, CMS}

\date{February 16, 2026}

\begin{document}

\frame{\titlepage}

\begin{frame}{Overview: Samples, Goal, Methodology}
\begin{itemize}
  \item Samples:
  \begin{itemize}
    \item nominal MiNNLO: massless $b$, 5FS
    \item dedicated heavy $b$ quark Z$b\bar{b}$ MiNNLO: massive $b$, 4FS. (From arXiv2404.08598.)
  \end{itemize}
  \item Using $\sigma=25.28$pb from Table 1 in arXiv2404.08598, but unclear if that is inclusive $\ell=e,\mu$ or for each lepton flavor.
  \item Goal: build a $b$-mass uncertainty by replacing selected nominal MiNNLO events with an equivalent massive-$b$-quark MiNNLO component.
  \item Method: select events with at least one gen-level $B$ hadron and build the corrected distribution as nominal MiNNLO (unselected) + massive-$b$ MiNNLO (selected).
\end{itemize}
\end{frame}

\begin{frame}{B-Hadron Diagnostics}
\begin{itemize}
  \item B hadrons are identified from GenPart using PDG-ID hadron-content logic (contains bottom flavor) with GenPart status 1 or 2.
  \item These plots use the $p_T>5$ B-hadron observables to exclude events with no reconstructed B hadrons in the shown distributions.
  \item Exact implementation is shown in backup.
\end{itemize}
\begin{columns}[T,onlytextwidth]
\column{0.5\textwidth}
\includegraphics[width=0.95\linewidth]{\detokenize{/home/submit/lavezzo/public_html/alphaS/260216_z_bb/hadrons/bhad_ge1_1730_rerun_260216_1745/samples_comparison_leadB_pt5.png}}
\column{0.5\textwidth}
\includegraphics[width=0.95\linewidth]{\detokenize{/home/submit/lavezzo/public_html/alphaS/260216_z_bb/hadrons/bhad_ge1_1730_rerun_260216_1745/samples_comparison_subB_pt5.png}}
\end{columns}
\end{frame}

\begin{frame}{Swap Result and Definition}
\begin{itemize}
  \item Swap definition: select events with at least one such $B$ hadron; corrected distribution is built as nominal MiNNLO (unselected) + massive-$b$ MiNNLO (selected).
\end{itemize}
\begin{center}
\includegraphics[width=0.62\textwidth,height=0.42\textheight,keepaspectratio]{\detokenize{/home/submit/lavezzo/public_html/alphaS/260216_z_bb/hadrons/bhad_ge1_1730_rerun_260216_1745/inclusive_ptVgen.png}}
\end{center}
\begin{itemize}
  \item \small Result: swapping gives a sizable normalization shift (not shape-only), so it is too aggressive as-is.
\end{itemize}
\end{frame}

\begin{frame}{Backup}

\end{frame}

\begin{frame}{Backup: LHE $b\bar{b}$ Kinematics}
\begin{itemize}
  \item \small LHE $b$-quark observables built with $|\mathrm{pdgId}|=5$ and final-state status criteria in the histmaker.
\end{itemize}
\begin{columns}[T,onlytextwidth]
\column{0.5\textwidth}
\includegraphics[width=0.95\linewidth]{\detokenize{/home/submit/lavezzo/public_html/alphaS/260216_z_bb/hadrons/bhad_ge1_1730_rerun_260216_1745/samples_comparison_m_bb_lhe.png}}
\column{0.5\textwidth}
\includegraphics[width=0.95\linewidth]{\detokenize{/home/submit/lavezzo/public_html/alphaS/260216_z_bb/hadrons/bhad_ge1_1730_rerun_260216_1745/samples_comparison_dR_bb_lhe.png}}
\end{columns}
\end{frame}

\begin{frame}{Backup: LHE Multiplicity (Initial vs Final)}
\begin{itemize}
  \item \small Requested LHE multiplicities: number of $b$ quarks in initial and final state.
\end{itemize}
\begin{columns}[T,onlytextwidth]
\column{0.5\textwidth}
\includegraphics[width=0.95\linewidth]{\detokenize{/home/submit/lavezzo/public_html/alphaS/260216_z_bb/hadrons/bhad_ge1_1730_rerun_260216_1745/samples_comparison_n_lhe_init_bbbar.png}}
\column{0.5\textwidth}
\includegraphics[width=0.95\linewidth]{\detokenize{/home/submit/lavezzo/public_html/alphaS/260216_z_bb/hadrons/bhad_ge1_1730_rerun_260216_1745/samples_comparison_n_lhe_fin_bbbar.png}}
\end{columns}
\end{frame}

\begin{frame}{Backup: LHE Leading-$b$ $p_T$ and Total Multiplicity}
\begin{itemize}
  \item \small Requested LHE leading-$b$ observable: final-state maximum $b$-quark $p_T$.
\end{itemize}
\begin{columns}[T,onlytextwidth]
\column{0.5\textwidth}
\includegraphics[width=0.95\linewidth]{\detokenize{/home/submit/lavezzo/public_html/alphaS/260216_z_bb/hadrons/bhad_ge1_1730_rerun_260216_1745/samples_comparison_lhe_bbbar_fin_max_pt.png}}
\column{0.5\textwidth}
\includegraphics[width=0.95\linewidth]{\detokenize{/home/submit/lavezzo/public_html/alphaS/260216_z_bb/hadrons/bhad_ge1_1730_rerun_260216_1745/samples_comparison_n_lhe_bbbar.png}}
\end{columns}
\end{frame}

\begin{frame}{Backup: GenJet $b$-jet Observables}
\begin{itemize}
  \item \small Gen $b$-jets are defined with hadronFlavour = 5; multiplicity uses the standard jet-threshold selection in this study setup.
\end{itemize}
\begin{columns}[T,onlytextwidth]
\column{0.5\textwidth}
\includegraphics[width=0.95\linewidth]{\detokenize{/home/submit/lavezzo/public_html/alphaS/260216_z_bb/hadrons/bhad_ge1_1730_rerun_260216_1745/samples_comparison_n_bjets.png}}
\column{0.5\textwidth}
\includegraphics[width=0.95\linewidth]{\detokenize{/home/submit/lavezzo/public_html/alphaS/260216_z_bb/hadrons/bhad_ge1_1730_rerun_260216_1745/samples_comparison_lead_bjet_pt.png}}
\end{columns}
\end{frame}

\begin{frame}{Backup: GenJet $\Delta R_{bb}$ and $m_{bb}$}
\begin{itemize}
  \item \small Requested jet-level pair observables from gen $b$-jets.
\end{itemize}
\begin{columns}[T,onlytextwidth]
\column{0.5\textwidth}
\includegraphics[width=0.95\linewidth]{\detokenize{/home/submit/lavezzo/public_html/alphaS/260216_z_bb/hadrons/bhad_ge1_1730_rerun_260216_1745/samples_comparison_dR_bb_jet.png}}
\column{0.5\textwidth}
\includegraphics[width=0.95\linewidth]{\detokenize{/home/submit/lavezzo/public_html/alphaS/260216_z_bb/hadrons/bhad_ge1_1730_rerun_260216_1745/samples_comparison_m_bb_jet.png}}
\end{columns}
\end{frame}

\begin{frame}[fragile]{Backup: Exact B-hadron Selection Code}
\begin{block}{Implementation}
\begin{verbatim}
// Source of truth: external library helper (ThePEG::PDT, from its .hpp)
// Treat this external helper as authoritative for B-hadron ID
if (status[i] != 1 && status[i] != 2) continue;
const int apdg = std::abs(pdgId[i]);
if (!ThePEG::PDT::hasBottom(apdg)) continue;
idx.push_back((int)i);

// Histmaker usage
df = df.Define("bHadIdx", "wrem::finalStateBHadronIdx(GenPart_pdgId, GenPart_status)")
df = df.Define("bHad_pt", "Take(GenPart_pt, bHadIdx)")
df = df.Define("nBhad", "static_cast<int>(bHad_pt.size())")
df = df.Define("bottom_sel", "(bHad_pt.size() >= 1)")
\end{verbatim}
\end{block}
\end{frame}

\begin{frame}{Backup: Unnormalized Swap with 4FS Scaled by 2}
\begin{itemize}
  \item \small Same unnormalized swap definition as main slide, but with an extra factor-2 scaling applied to the selected 4FS component before replacement.
\end{itemize}
\begin{center}
\includegraphics[width=0.62\textwidth,height=0.42\textheight,keepaspectratio]{\detokenize{/home/submit/lavezzo/public_html/alphaS/260216_z_bb/hadrons/bhad_ge1_1730_rerun_260216_1745/inclusive_ptVgen_unnorm_massiveX2_focus.png}}
\end{center}
\end{frame}

\end{document}
