\documentclass[10pt]{beamer}

\usetheme{Madrid}

\usepackage{graphicx}

\usepackage{booktabs}

\title{Z b-mass Uncertainty Study}

\subtitle{AlphaS Group Meeting}

\author{Luca}

\date{February 14, 2026}

\begin{document}

\frame{\titlepage}

\begin{frame}{Goal and Strategy}
\begin{itemize}
  \item Goal: derive a nuisance for MiNNLO Z (5FS, massless b quarks) using a comparison to Zbb MiNNLO (4FS, massive b quarks).
  \item Core method: compare distributions between the two samples and interpret differences as candidate nuisance content.
  \item Important context: 5FS vs 4FS scheme differences are part of the physical effect entering this nuisance.
\end{itemize}
\end{frame}

\begin{frame}{Samples and Object Definitions}
\begin{itemize}
  \item Nominal sample: inclusive Zmumu MiNNLO (5FS, massless b).
  \item Alternate sample: Zbb MiNNLO (4FS, massive b). (Note: it seems this sample is actually $Z \to ee$.)
  \item Objects shown in this draft: LHE bb observables, gen b-jet observables, and B-hadron observables from GenPart-based definitions in the histmaker.
\end{itemize}
\end{frame}

\begin{frame}{LHE Composition}
\begin{itemize}
  \item \small Counts are built from LHEPart with b quarks identified by $|\mathrm{pdgId}|=5$ and split by status: initial-state has status $=-1$, final-state has status $=1$.
\end{itemize}
\begin{columns}[T,onlytextwidth]
\column{0.5\textwidth}
\includegraphics[width=0.95\linewidth]{\detokenize{/home/submit/lavezzo/public_html/alphaS/260214_z_bb/hadrons/bhad_allvars_260214_labels/samples_comparison_n_lhe_init_bbbar.png}}
\column{0.5\textwidth}
\includegraphics[width=0.95\linewidth]{\detokenize{/home/submit/lavezzo/public_html/alphaS/260214_z_bb/hadrons/bhad_allvars_260214_labels/samples_comparison_n_lhe_fin_bbbar.png}}
\end{columns}
\end{frame}

\begin{frame}{LHE Kinematics}
\begin{itemize}
  \item \small $m_{bb}^{\mathrm{LHE}}$ and $\Delta R_{bb}^{\mathrm{LHE}}$ are computed from LHE $b$ and $\bar b$ quarks with $|\mathrm{pdgId}|=5$; $p_{T}$ observables use final-state LHE $b$ quarks with status $=1$.
\end{itemize}
\begin{columns}[T,onlytextwidth]
\column{0.5\textwidth}
\includegraphics[width=0.95\linewidth]{\detokenize{/home/submit/lavezzo/public_html/alphaS/260214_z_bb/hadrons/bhad_allvars_260214_labels/samples_comparison_m_bb_lhe.png}}
\column{0.5\textwidth}
\includegraphics[width=0.95\linewidth]{\detokenize{/home/submit/lavezzo/public_html/alphaS/260214_z_bb/hadrons/bhad_allvars_260214_labels/samples_comparison_dR_bb_lhe.png}}
\end{columns}
\begin{itemize}
  \item \small Swapping at LHE-quark level is unphysical here because the two samples differ by flavor scheme (5FS vs 4FS), not just by a small kinematic perturbation.
\end{itemize}
\end{frame}

\begin{frame}{LHE b-Quark pT Spectra}
\begin{itemize}
  \item \small Final-state LHE $b$ quarks are selected with status $=1$ and $|\mathrm{pdgId}|=5$; shown are event-wise minimum and maximum $p_{T}$ across final-state $b/\bar b$ quarks.
\end{itemize}
\begin{columns}[T,onlytextwidth]
\column{0.5\textwidth}
\includegraphics[width=0.95\linewidth]{\detokenize{/home/submit/lavezzo/public_html/alphaS/260214_z_bb/hadrons/bhad_allvars_260214_labels/samples_comparison_lhe_bbbar_fin_min_pt.png}}
\column{0.5\textwidth}
\includegraphics[width=0.95\linewidth]{\detokenize{/home/submit/lavezzo/public_html/alphaS/260214_z_bb/hadrons/bhad_allvars_260214_labels/samples_comparison_lhe_bbbar_fin_max_pt.png}}
\end{columns}
\end{frame}

\begin{frame}{Jet-Level Composition and pT}
\begin{itemize}
  \item \small A gen $b$-jet is defined by $\mathrm{hadronFlavour}=5$; $n_{\mathrm{bjets}}$ uses $p_{T}>20\,\mathrm{GeV}$ and $|\eta|<2.5$, while the leading-jet $p_{T}$ panel uses the loose ordered list with $p_{T}>0$ and $|\eta|<10$ (as stored in NanoAOD).
\end{itemize}
\begin{columns}[T,onlytextwidth]
\column{0.5\textwidth}
\includegraphics[width=0.95\linewidth]{\detokenize{/home/submit/lavezzo/public_html/alphaS/260214_z_bb/hadrons/bhad_allvars_260214_labels/samples_comparison_n_bjets.png}}
\column{0.5\textwidth}
\includegraphics[width=0.95\linewidth]{\detokenize{/home/submit/lavezzo/public_html/alphaS/260214_z_bb/hadrons/bhad_allvars_260214_labels/samples_comparison_lead_bjet_pt.png}}
\end{columns}
\begin{itemize}
  \item \small Because gen jets in NanoAOD effectively have a threshold near $20\,\mathrm{GeV}$, many events in the Zbb sample still have no tagged $b$-jets, so this is not a robust swap handle.
\end{itemize}
\end{frame}

\begin{frame}{Jet-Level bb Pair Kinematics}
\begin{itemize}
  \item \small $m_{bb}^{\mathrm{jet}}$ and $\Delta R_{bb}^{\mathrm{jet}}$ are computed from the leading two gen $b$-jets after requiring $p_{T}>20\,\mathrm{GeV}$ and $|\eta|<2.5$.
\end{itemize}
\begin{columns}[T,onlytextwidth]
\column{0.5\textwidth}
\includegraphics[width=0.95\linewidth]{\detokenize{/home/submit/lavezzo/public_html/alphaS/260214_z_bb/hadrons/bhad_allvars_260214_labels/samples_comparison_m_bb_jet.png}}
\column{0.5\textwidth}
\includegraphics[width=0.95\linewidth]{\detokenize{/home/submit/lavezzo/public_html/alphaS/260214_z_bb/hadrons/bhad_allvars_260214_labels/samples_comparison_dR_bb_jet.png}}
\end{columns}
\begin{itemize}
  \item \small The ratios are relatively flat between the two samples in these jet-pair observables.
\end{itemize}
\end{frame}

\begin{frame}{B-hadron Multiplicity and pT}
\begin{itemize}
  \item \small $B$ hadrons are selected from final-state GenPart objects using the $B$-hadron identifier; multiplicity and $p_{T}$ observables on this slide require hadron $p_{T}>5\,\mathrm{GeV}$.
\end{itemize}
\begin{columns}[T,onlytextwidth]
\column{0.5\textwidth}
\includegraphics[width=0.95\linewidth]{\detokenize{/home/submit/lavezzo/public_html/alphaS/260214_z_bb/hadrons/bhad_allvars_260214_labels/samples_comparison_nBhad_pt5.png}}
\column{0.5\textwidth}
\includegraphics[width=0.95\linewidth]{\detokenize{/home/submit/lavezzo/public_html/alphaS/260214_z_bb/hadrons/bhad_allvars_260214_labels/samples_comparison_subB_pt5.png}}
\end{columns}
\begin{itemize}
  \item \small Ratios are fairly flat in shape, but there is a sizeable normalization difference; unnormalized swapping would induce an overly large systematic, so a normalized-swap option is motivated.
\end{itemize}
\end{frame}

\begin{frame}{B-hadron Pair Kinematics}
\begin{itemize}
  \item \small This slide is restricted to events with at least two $B$ hadrons above $5\,\mathrm{GeV}$; pair observables use the leading two selected $B$ hadrons.
\end{itemize}
\begin{columns}[T,onlytextwidth]
\column{0.5\textwidth}
\includegraphics[width=0.95\linewidth]{\detokenize{/home/submit/lavezzo/public_html/alphaS/260214_z_bb/hadrons/bhad_allvars_260214_labels/samples_comparison_m_bb_had.png}}
\column{0.5\textwidth}
\includegraphics[width=0.95\linewidth]{\detokenize{/home/submit/lavezzo/public_html/alphaS/260214_z_bb/hadrons/bhad_allvars_260214_labels/samples_comparison_dR_bb_had.png}}
\end{columns}
\end{frame}

\begin{frame}{Swap Procedure on $p_{T}^{V,\mathrm{gen}}$ (Unnormalized)}
\begin{itemize}
  \item \small Procedure: select events with at least two $B$ hadrons above $5\,\mathrm{GeV}$ and subleading $B$-hadron $p_T>10\,\mathrm{GeV}$ in both samples; replace the selected 5FS component with the selected 4FS component without extra scaling.
\end{itemize}
\begin{center}
\includegraphics[width=0.62\textwidth,height=0.42\textheight,keepaspectratio]{\detokenize{/home/submit/lavezzo/public_html/alphaS/260214_z_bb/hadrons/bhad_allvars_parton_260214_50thr/inclusive_ptVgen.png}}
\end{center}
\begin{itemize}
  \item \small Conclusion: this direct swap induces a too-large systematic because of the normalization difference.
\end{itemize}
\end{frame}

\begin{frame}{Swap Procedure on $p_{T}^{V,\mathrm{gen}}$ (Normalized)}
\begin{itemize}
  \item \small Procedure: same event selection as previous slide, but normalize the selected 4FS component to the selected 5FS yield before replacement.
\end{itemize}
\begin{center}
\includegraphics[width=0.62\textwidth,height=0.42\textheight,keepaspectratio]{\detokenize{/home/submit/lavezzo/public_html/alphaS/260214_z_bb/hadrons/bhad_allvars_parton_260214_50thr_norm/inclusive_ptVgen_normalized.png}}
\end{center}
\begin{itemize}
  \item \small With normalization, the ratio is much flatter and the effect is more shape-like, making this prescription more suitable as a nuisance candidate, though it may be unphysical.
\end{itemize}
\end{frame}

\begin{frame}{Backup: Parton-Flavour Jet Multiplicity and Leading $p_T$}
\begin{itemize}
  \item \small Backup check: define jet tags with parton flavour ($|\mathrm{partonFlavour}|=5$) and use the same jet kinematic cuts as in the hadron-flavour study.
\end{itemize}
\begin{columns}[T,onlytextwidth]
\column{0.5\textwidth}
\includegraphics[width=0.95\linewidth]{\detokenize{/home/submit/lavezzo/public_html/alphaS/260214_z_bb/hadrons/bhad_allvars_parton_260214_50thr/samples_comparison_n_bjets_parton.png}}
\column{0.5\textwidth}
\includegraphics[width=0.95\linewidth]{\detokenize{/home/submit/lavezzo/public_html/alphaS/260214_z_bb/hadrons/bhad_allvars_parton_260214_50thr/samples_comparison_lead_bjet_pt_parton.png}}
\end{columns}
\begin{itemize}
  \item \small These parton-flavour-tagged shapes are qualitatively very similar to the hadron-flavour-tagged ones.
\end{itemize}
\end{frame}

\begin{frame}{Backup: Parton-Flavour Jet Pair Kinematics}
\begin{itemize}
  \item \small Backup check: compare $m_{bb}^{\mathrm{jet}}$ and $\Delta R_{bb}^{\mathrm{jet}}$ built from parton-flavour-tagged jets.
\end{itemize}
\begin{columns}[T,onlytextwidth]
\column{0.5\textwidth}
\includegraphics[width=0.95\linewidth]{\detokenize{/home/submit/lavezzo/public_html/alphaS/260214_z_bb/hadrons/bhad_allvars_parton_260214_50thr/samples_comparison_m_bb_jet_parton.png}}
\column{0.5\textwidth}
\includegraphics[width=0.95\linewidth]{\detokenize{/home/submit/lavezzo/public_html/alphaS/260214_z_bb/hadrons/bhad_allvars_parton_260214_50thr/samples_comparison_dR_bb_jet_parton.png}}
\end{columns}
\begin{itemize}
  \item \small Again, behavior is close to the hadron-flavour case, so this does not materially change the swap-handle conclusion.
\end{itemize}
\end{frame}

\end{document}
