\documentclass[10pt]{beamer}

\usetheme{Madrid}

\usepackage{graphicx}

\usepackage{booktabs}

\title{Z $b$-mass Uncertainty Study}

\subtitle{3-Slide Status Summary + Backups}

\author{Luca}

\date{February 16, 2026}

\begin{document}

\frame{\titlepage}

\begin{frame}{Overview: Samples, Goal, Methodology}
\begin{itemize}
  \item Samples: 5FS nominal = Zmumu MiNNLO (massless $b$), 4FS alternate = Zbb MiNNLO (massive $b$).
  \item Goal: build a $b$-mass uncertainty nuisance by replacing a selected 5FS heavy-flavor component with the 4FS component.
  \item Method: define a swap region using B hadrons from GenPart (status 1 or 2), compare object-level and inclusive-shape impacts, and assess unnormalized swap behavior first.
\end{itemize}
\end{frame}

\begin{frame}{B-Hadron Diagnostics (No B-hadron $p_T$ Cut)}
\begin{itemize}
  \item \small B hadrons are identified from GenPart using PDG-ID hadron-content logic (contains bottom flavor) and requiring GenPart status 1 or 2; no additional B-hadron $p_T$ threshold is applied for these two observables. Exact implementation is shown in backup slide `Backup: Exact B-hadron Selection Code`.
\end{itemize}
\begin{columns}[T,onlytextwidth]
\column{0.5\textwidth}
\includegraphics[width=0.95\linewidth]{\detokenize{/home/submit/lavezzo/public_html/alphaS/260216_z_bb/hadrons/bhad_ge1_260216_50thr/samples_comparison_leadB_pt_focus.png}}
\column{0.5\textwidth}
\includegraphics[width=0.95\linewidth]{\detokenize{/home/submit/lavezzo/public_html/alphaS/260216_z_bb/hadrons/bhad_ge1_260216_50thr/samples_comparison_nBhad_focus.png}}
\end{columns}
\end{frame}

\begin{frame}{Unnormalized Swap Result and Definition}
\begin{itemize}
  \item \small B-hadron definition: GenPart object passing the same bottom-hadron PDG-ID logic with status 1 or 2. Swap definition: select events with at least one such $B$ hadron ($nBhad \geq 1$); corrected distribution is built as 5FS (unselected) + 4FS (selected), with no normalization factor.
\end{itemize}
\begin{center}
\includegraphics[width=0.62\textwidth,height=0.42\textheight,keepaspectratio]{\detokenize{/home/submit/lavezzo/public_html/alphaS/260216_z_bb/hadrons/bhad_ge1_260216_50thr/inclusive_ptVgen_unnorm_focus.png}}
\end{center}
\begin{itemize}
  \item \small Result: unnormalized swapping gives a sizable normalization shift (not shape-only), so it is too aggressive as-is.
\end{itemize}
\end{frame}

\begin{frame}{Backup}

\end{frame}

\begin{frame}{Backup: LHE $b\bar{b}$ Kinematics}
\begin{itemize}
  \item \small LHE $b$-quark observables built with $|\mathrm{pdgId}|=5$ and final-state status criteria in the histmaker.
\end{itemize}
\begin{columns}[T,onlytextwidth]
\column{0.5\textwidth}
\includegraphics[width=0.95\linewidth]{\detokenize{/home/submit/lavezzo/public_html/alphaS/260216_z_bb/hadrons/bhad_ge1_260216_50thr/samples_comparison_m_bb_lhe_focus.png}}
\column{0.5\textwidth}
\includegraphics[width=0.95\linewidth]{\detokenize{/home/submit/lavezzo/public_html/alphaS/260216_z_bb/hadrons/bhad_ge1_260216_50thr/samples_comparison_dR_bb_lhe_focus.png}}
\end{columns}
\end{frame}

\begin{frame}{Backup: GenJet $b$-jet Observables}
\begin{itemize}
  \item \small Gen $b$-jets are defined with hadronFlavour = 5; multiplicity uses the standard jet-threshold selection in this study setup.
\end{itemize}
\begin{columns}[T,onlytextwidth]
\column{0.5\textwidth}
\includegraphics[width=0.95\linewidth]{\detokenize{/home/submit/lavezzo/public_html/alphaS/260216_z_bb/hadrons/bhad_ge1_260216_50thr/samples_comparison_n_bjets_focus.png}}
\column{0.5\textwidth}
\includegraphics[width=0.95\linewidth]{\detokenize{/home/submit/lavezzo/public_html/alphaS/260216_z_bb/hadrons/bhad_ge1_260216_50thr/samples_comparison_lead_bjet_pt_focus.png}}
\end{columns}
\end{frame}

\begin{frame}[fragile]{Backup: Exact B-hadron Selection Code}
\begin{block}{Cleaned Implementation}
\begin{verbatim}
// Source of truth: wremnants/include/theoryTools.hpp
// Use the library helper as authoritative (ignore local implementation differences)
if (status[i] != 1 && status[i] != 2) continue;
const int apdg = std::abs(pdgId[i]);
if (!isBHadron(apdg)) continue;
idx.push_back((int)i);

// Histmaker usage (w_z_gen_dists.py)
df = df.Define("bHadIdx", "wrem::finalStateBHadronIdx(GenPart_pdgId, GenPart_status)")
df = df.Define("bHad_pt", "Take(GenPart_pt, bHadIdx)")
df = df.Define("nBhad", "static_cast<int>(bHad_pt.size())")
df = df.Define("bottom_sel", "(bHad_pt.size() >= 1)")
\end{verbatim}
\end{block}
\end{frame}

\begin{frame}{Backup: Unnormalized Swap with 4FS Scaled by 2}
\begin{itemize}
  \item \small Same unnormalized swap definition as main slide, but with an extra factor-2 scaling applied to the selected 4FS component before replacement.
\end{itemize}
\begin{center}
\includegraphics[width=0.62\textwidth,height=0.42\textheight,keepaspectratio]{\detokenize{/home/submit/lavezzo/public_html/alphaS/260216_z_bb/hadrons/bhad_ge1_260216_50thr/inclusive_ptVgen_unnorm_massiveX2_focus.png}}
\end{center}
\end{frame}

\end{document}
